% In this file you should put the actual content of the blueprint.
% It will be used both by the web and the print version.
% It should *not* include the \begin{document}
%
% If you want to split the blueprint content into several files then
% the current file can be a simple sequence of \input. Otherwise It
% can start with a \section or \chapter for instance.

\section{Introduction}
The Cauchy-Davenport Theorem, which has numerous applications in Additive Number Theory, is the following.

\begin{theorem}[Cauchy Davenport Theorem\cite{3}]
If $p$ is a prime, and $A, B$ are two nonempty subsets of $\mathbb{Z}_p$, then
\[
|A + B| \geq \min\{p, |A| + |B| - 1\}.
\]
\end{theorem}

This theorem can be proved quickly by induction on $|B|$. A different proof has recently been found by the authors \cite{1}. This proof is based on a simple algebraic technique, and its main advantage is that it extends easily and gives several related results. Some of the simplest results are described in \cite{1}. In the present paper we describe the general technique and apply it to deduce various additional consequences. A representative example is the following.

\begin{proposition} \label{prop:main-example}
Let $p$ be a prime, and let $A_0, A_1, \ldots, A_k$ be nonempty subsets of the cyclic group $\mathbb{Z}_p$. If $|A_i| \neq |A_j|$ for all $0 \leq i < j \leq k$ and $\sum_{i=0}^k |A_i| \leq p + \binom{k+2}{2} - 1$ then
\[
|\{a_0 + a_1 + \cdots + a_k : a_i \in A_i, \; a_i \neq a_j \text{ for all } i \neq j\}| 
\geq \sum_{i=0}^k |A_i| - \binom{k+2}{2} + 1.
\]
\end{proposition}

Note that the very special case of this proposition in which $k=1$, $A_0 = A$ and $A_1 = A \setminus \{a\}$ for an arbitrary element $a \in A$ implies that if $A \subset \mathbb{Z}_p$ and $2|A| - 1 \leq p + 2$ then the number of sums $a_1 + a_2$ with $a_1, a_2 \in A$ and $a_1 \neq a_2$ is at least $2|A| - 3$. This easily implies the following theorem, conjectured by Erd\H{o}s and Heilbronn in 1964 (cf., e.g., \cite{5}) and proved very recently by Dias Da Silva and Hamidoune \cite{4}, using some tools from linear algebra and the representation theory of the symmetric group.

\begin{theorem}[\cite{4}] \label{thm:ErdosHeilbronn}
If $p$ is a prime, and $A$ is a nonempty subset of $\mathbb{Z}_p$, then
\[
|\{a + a' : a, a' \in A, \; a \neq a'\}| \geq \min\{p, 2|A| - 3\}.
\]
\end{theorem}

The rest of the paper is organized as follows. In Section~2 we present and prove a general result and show how it implies the Cauchy-Davenport theorem. In Section~3 we consider the addition of distinct residues and prove Proposition~\ref{prop:main-example} and some of its consequences. Section~4 contains some further applications of the general theorem and the final Section~5 concludes with various remarks and open problems.

\section{The General Theorem}
Let $p$ be a prime. For a polynomial $h = h(x_0, x_1, \ldots, x_k)$ over $\mathbb{Z}_p$ and for subsets $A_0, A_1, \ldots, A_k$ of $\mathbb{Z}_p$, define
\[
\oplus_h \sum_{i=0}^k A_i = \{a_0 + a_1 + \cdots + a_k : a_i \in A_i, \; h(a_0, a_1, \ldots, a_k) \neq 0\}.
\]

Our main tool is the following.

\begin{theorem} \label{thm:general}
Let $p$ be a prime and let $h = h(x_0, \ldots, x_k)$ be a polynomial over $\mathbb{Z}_p$. Let $A_0, A_1, \ldots, A_k$ be nonempty subsets of $\mathbb{Z}_p$, where $|A_i| = c_i + 1$ and define $m = \sum_{i=0}^k c_i - \deg(h)$. If the coefficient of $\prod_{i=0}^k x_i^{c_i}$ in
\[
(x_0 + x_1 + \cdots + x_k)^m h(x_0, x_1, \ldots, x_k)
\]
is nonzero (in $\mathbb{Z}_p$) then
\[
\left| \oplus_h \sum_{i=0}^k A_i \right| \geq m + 1
\]
(and hence $m < p$).
\end{theorem}

In order to prove this theorem we need the following simple and well known lemma, which is proved in various places (see, e.g., \cite{2}). Since the argument is very short we reproduce it here.

\begin{lemma} \label{lem:polynomial-vanishing}
Let $P = P(x_0, x_1, \ldots, x_k)$ be a polynomial in $k+1$ variables over an arbitrary field $F$. Suppose that the degree of $P$ as a polynomial in $x_i$ is at most $c_i$ for $0 \leq i \leq k$, and let $A_i \subset F$ be a set of cardinality $c_i + 1$. If $P(x_0, x_1, \ldots, x_k) = 0$ for all $(k+1)$-tuples $(x_0, \ldots, x_k) \in A_0 \times A_1 \times \cdots \times A_k$, then $P \equiv 0$, that is: all the coefficients in $P$ are zeros.
\end{lemma}

\begin{proof}
We apply induction on $k$. For $k = 0$, the lemma is simply the assertion that a non-zero polynomial of degree $c_0$ in one variable can have at most $c_0$ distinct zeros. Assuming that the lemma holds for $k-1$, we prove it for $k$ ($k \geq 1$). Given a polynomial $P = P(x_0, \ldots, x_k)$ and sets $A_i$ satisfying the hypotheses of the lemma, let us write $P$ as a polynomial in $x_k$, that is,
\[
P = \sum_{i=0}^{c_k} P_i(x_0, \ldots, x_{k-1}) x_k^i,
\]
where each $P_i$ is a polynomial with $x_j$-degree bounded by $c_j$. For each fixed $k$-tuple $(x_0, \ldots, x_{k-1}) \in A_0 \times A_1 \times \cdots \times A_{k-1}$, the polynomial in $x_k$ obtained from $P$ by substituting the values of $x_0, \ldots, x_{k-1}$ vanishes for all $x_k \in A_k$, and is thus identically $0$. Thus $P_i(x_0, \ldots, x_{k-1}) = 0$ for all $(x_0, \ldots, x_{k-1}) \in A_0 \times \cdots \times A_{k-1}$. Hence, by the induction hypothesis, $P_i \equiv 0$ for all $i$, implying that $P \equiv 0$. This completes the induction and the proof of the lemma.
\end{proof}

\begin{proof}[Proof of Theorem~\ref{thm:general}]
Suppose the assertion is false, and let $E$ be a (multi-)set of $m$ (not necessarily distinct) elements of $\mathbb{Z}_p$ that contains the set $\oplus_h \sum_{i=0}^k A_i$. Let $Q = Q(x_0, \ldots, x_k)$ be the polynomial defined as follows:
\[
Q(x_0, \ldots, x_k) = h(x_0, x_1, \ldots, x_k) \cdot \prod_{e \in E} (x_0 + \cdots + x_k - e).
\]
Note that
\[
Q(x_0, \ldots, x_k) = 0 \quad \text{for all } (x_0, \ldots, x_k) \in A_0 \times \cdots \times A_k. \tag{1}
\]
This is because for each such $(x_0, \ldots, x_k)$ either $h(x_0, \ldots, x_k) = 0$ or $x_0 + \cdots + x_k \in \oplus_h \sum_{i=0}^k A_i \subset E$. Note also that $\deg(Q) = m + \deg(h) = \sum_{i=0}^k c_i$ and hence the coefficient of the monomial $x_0^{c_0} \cdots x_k^{c_k}$ in $Q$ is the same as that of this monomial in the polynomial $(x_0 + \cdots + x_k)^m h(x_0, \ldots, x_k)$, which is nonzero, by assumption.

For each $i$, $0 \leq i \leq k$, define
\[
g_i(x_i) = \prod_{a \in A_i} (x_i - a) = x_i^{c_i + 1} - \sum_{j=0}^{c_i} b_{ij} x_i^j.
\]
Let $\overline{Q} = \overline{Q}(x_0, \ldots, x_k)$ be the polynomial obtained from the standard representation of $Q$ as a linear combination of monomials by replacing, repeatedly, each occurrence of $x_i^{c_i + 1}$ by $\sum_{j=0}^{c_i} b_{ij} x_i^j$. Note that since for every $x_i \in A_i$, $x_i^{c_i + 1}$ is equal to this sum, equation (1) holds for $\overline{Q}$ as well. However, the $x_i$-degree of $\overline{Q}$ is at most $c_i$ and hence, by Lemma~\ref{lem:polynomial-vanishing} it is identically zero. To obtain a contradiction, we claim that the coefficient of the monomial $\prod_{i=0}^k x_i^{c_i}$ in $\overline{Q}$ is not $0$ (in $\mathbb{Z}_p$). To see this note that the coefficient of this monomial in $Q$ is nonzero modulo $p$ by assumption. The crucial observation is that the coefficient of this monomial in $\overline{Q}$ is equal to its coefficient in $Q$. This is because the process of replacing each of the expressions $x_i^{c_i + 1}$ by $\sum_{j=0}^{c_i} b_{ij} x_i^j$ does not affect the above monomial itself. Moreover, since the total degree of $Q$ is $\sum_{i=0}^k c_i$ and the process of replacing the expressions as above strictly reduces degrees, this process cannot create any additional scalar multiples of this monomial, proving the claim.

It thus follows that $\overline{Q}$ is not identically zero, supplying the desired contradiction and completing the proof.
\end{proof}

The simplest application of Theorem~\ref{thm:general} is the following proof of the Cauchy-Davenport Theorem (Theorem~\ref{thm:ErdosHeilbronn}).

\begin{proof}[Proof of Theorem~\ref{thm:ErdosHeilbronn}]
If $|A| + |B| \leq p + 1$ apply Theorem~\ref{thm:general} with $h \equiv 1$, $k = 1$, $A_0 = A$, $A_1 = B$ and $m = |A| + |B| - 2$. Here $c_0 = |A| - 1$, $c_1 = |B| - 1$ and the relevant coefficient is $\binom{m}{c_0}$ which is nonzero modulo $p$ (as $m < p$). If $|A| + |B| > p + 1$ simply replace $B$ by a subset $B'$ of cardinality $p + 1 - |A|$ and apply the result above to $A$ and $B'$ to conclude that in this case $|A + B| \geq |A + B'| = p$.
\end{proof}

\section{Adding Distinct Residues}
The following Lemma can be easily deduced from the known results about the Ballot problem (see, e.g., \cite{8}), as well as from the known connection between this problem and the hook formula for the number of Young tableaux of a given shape. Here we present a simple, self contained proof.

\begin{lemma} \label{lem:coefficient-vandermonde}
Let $c_0, \ldots, c_k$ be nonnegative integers and suppose that $\sum_{i=0}^k c_i = m + \binom{k+1}{2}$, where $m$ is a nonnegative integer. Then the coefficient of $\prod_{i=0}^k x_i^{c_i}$ in the polynomial
\[
(x_0 + x_1 + \cdots + x_k)^m \prod_{k \geq i > j \geq 0} (x_i - x_j)
\]
is
\[
\frac{m!}{c_0! c_1! \cdots c_k!} \prod_{k \geq i > j \geq 0} (c_i - c_j).
\]
\end{lemma}

\begin{proof}
The product $\prod_{k \geq i > j \geq 0} (x_i - x_j)$ is precisely the Vandermonde determinant $\det(x_i^j)_{0 \leq i \leq k, 0 \leq j \leq k}$ which is equal to the sum
\[
\sum_{\sigma \in S_{k+1}} (-1)^{\operatorname{sign}(\sigma)} \prod_{i=0}^k x_i^{\sigma(i)},
\]
where $S_{k+1}$ denotes the set of all permutations of the $k+1$ symbols $0, \ldots, k$. It thus follows that the required coefficient, which we denote by $C$, is given by
\[
C = \sum_{\sigma \in S_{k+1}} (-1)^{\operatorname{sign}(\sigma)} \frac{m!}{(c_0 - \sigma(0))! (c_1 - \sigma(1))! \cdots (c_k - \sigma(k))!}.
\]

Similarly, the product $\prod_{k \geq i > j \geq 0} (c_i - c_j)$ is the Vandermonde determinant $\det(c_i^j)_{0 \leq i \leq k, 0 \leq j \leq k}$. For two integers $r \geq 1$ and $s$ let $(s)_r$ denote the product $s(s-1)\cdots(s-r+1)$ and define also $(s)_0 = 1$ for all $s$. Observe that the matrix $((c_i)_j)_{0 \leq i \leq k, 0 \leq j \leq k}$ can be obtained from the matrix $(c_i^j)_{0 \leq i \leq k, 0 \leq j \leq k}$ by subtracting appropriate linear combinations of the columns with indices less than $j$ from the column indexed by $j$, for each $j = k, k-1, \ldots, 1$. Therefore, these two matrices have the same determinant. It thus follows that
\begin{align*}
\frac{m!}{c_0! c_1! \cdots c_k!} \prod_{k \geq i > j \geq 0} (c_i - c_j) 
&= \frac{m!}{c_0! c_1! \cdots c_k!} \det((c_i)_j)_{0 \leq i \leq k, 0 \leq j \leq k} \\
&= \frac{m!}{c_0! c_1! \cdots c_k!} \sum_{\sigma \in S_{k+1}} (-1)^{\operatorname{sign}(\sigma)} (c_0)_{\sigma(0)} (c_1)_{\sigma(1)} \cdots (c_k)_{\sigma(k)} \\
&= \sum_{\sigma \in S_{k+1}} (-1)^{\operatorname{sign}(\sigma)} \frac{m!}{(c_0 - \sigma(0))! (c_1 - \sigma(1))! \cdots (c_k - \sigma(k))!} = C,
\end{align*}
completing the proof.
\end{proof}

Let $p$ be a prime, and let $A_0, A_1, \ldots, A_k$ be nonempty subsets of the cyclic group $\mathbb{Z}_p$. Define
\[
\oplus_{i=0}^k A_i = \{a_0 + a_1 + \cdots + a_k : a_i \in A_i, \; a_i \neq a_j \text{ for all } i \neq j\}.
\]
In this notation, the assertion of Proposition~\ref{prop:main-example} is that if $|A_i| \neq |A_j|$ for all $0 \leq i < j \leq k$ and $\sum_{i=0}^k |A_i| \leq p + \binom{k+2}{2} - 1$ then
\[
\left| \oplus_{i=0}^k A_i \right| \geq \sum_{i=0}^k |A_i| - \binom{k+2}{2} + 1.
\]

\begin{proof}[Proof of Proposition~\ref{prop:main-example}]
Define
\[
h(x_0, \ldots, x_k) = \prod_{k \geq i > j \geq 0} (x_i - x_j),
\]
and note that for this $h$, the sum $\oplus_{i=0}^k A_i$ is precisely the sum $\oplus_h \sum_{i=0}^k A_i$. Suppose $|A_i| = c_i + 1$ and put
\[
m = \sum_{i=0}^k c_i - \binom{k+1}{2} \quad \left(= \sum_{i=0}^k |A_i| - \binom{k+2}{2}\right).
\]
By assumption $m < p$ and by Lemma~\ref{lem:coefficient-vandermonde} the coefficient of $\prod_{i=0}^k x_i^{c_i}$ in $h \cdot (x_0 + \cdots + x_k)^m$ is
\[
\frac{m!}{c_0! c_1! \cdots c_k!} \prod_{k \geq i > j \geq 0} (c_i - c_j),
\]
which is nonzero modulo $p$, since $m < p$ and the numbers $c_i$ are pairwise distinct. Since $m = \sum_{i=0}^k c_i - \deg(h)$, the desired result follows from Theorem~\ref{thm:general}.
\end{proof}

\begin{theorem} \label{thm:general-distinct-residues}
Let $p$ be a prime, and let $A_0, \ldots, A_k$ be nonempty subsets of $\mathbb{Z}_p$, where $|A_i| = b_i$, and suppose $b_0 \geq b_1 \geq \cdots \geq b_k$. Define $b_0', \ldots, b_k'$ by
\[
b_0' = b_0 \quad \text{and} \quad b_i' = \min\{b_{i-1}' - 1, b_i\}, \text{ for } 1 \leq i \leq k. \tag{2}
\]
If $b_k' > 0$ then
\[
\left| \oplus_{i=0}^k A_i \right| \geq \min\left\{p, \; \sum_{i=0}^k b_i' - \binom{k+2}{2} + 1\right\}.
\]
Moreover, the above estimate is sharp for all possible values of $p \geq b_0 \geq \cdots \geq b_k$.
\end{theorem}

\begin{proof}
If $b_i' \leq 0$ for some $i$ then $b_k' \leq 0$ and thus $b_i' > 0$ for all $i$. For each $i$, $1 \leq i \leq k$, let $A_i'$ be an arbitrary subset of cardinality $b_i'$ of $A_i$. Note that the cardinalities of the sets $A_i'$ are pairwise distinct and that $\oplus_{i=0}^k A_i' \subset \oplus_{i=0}^k A_i$. If $\sum_{i=0}^k b_i' \leq p + \binom{k+2}{2} - 1$ then
\[
\left| \oplus_{i=0}^k A_i \right| \geq \left| \oplus_{i=0}^k A_i' \right| \geq \sum_{i=0}^k b_i' - \binom{k+2}{2} + 1,
\]
by Proposition~\ref{prop:main-example}, as needed. Otherwise, we claim that there are $1 \leq b_k'' < b_{k-1}'' < \cdots < b_0''$, where $b_i'' \leq b_i'$ for all $i$ and $\sum_{i=0}^k b_i'' = p + \binom{k+2}{2} - 1$. To prove this claim, consider the operator $T$ that maps sequences of integers $(d_0, \ldots, d_k)$ with $d_0 > d_1 > \cdots > d_k \geq 1$ to sequences of the same kind defined as follows. The sequence $(k+1, \ldots, 1)$ is mapped to itself. For any other sequence $(d_0, \ldots, d_k)$, let $j$ be the largest index for which $d_j > k+1 - j$ and define $T(d_0, \ldots, d_k) = (d_0, \ldots, d_{j-1}, d_j - 1, d_{j+1}, \ldots, d_k)$. Clearly, the sum of the elements in $T(D)$ is one less than the sum of the elements of $D$ for every $D$ that differs from $(k+1, \ldots, 1)$, and thus, by repeatedly applying $T$ to our sequence $(b_0', \ldots, b_k')$ we get the desired sequence $(b_0'', \ldots, b_k'')$, proving the claim.

Returning to the proof of the theorem in case $\sum_{i=0}^k b_i' > p + \binom{k+2}{2} - 1$, let $b_i''$ be as in the claim, and apply Proposition~\ref{prop:main-example} to arbitrary subsets $A_i''$ of cardinality $b_i''$ of $A_i'$.

It remains to show that the estimate is best possible for all $p \geq b_0 \geq \cdots \geq b_k \geq 1$. This is shown by defining $A_i = \{1, 2, 3, \ldots, b_i\}$ for all $i$. It is easy to check that for these sets $A_i$, the set $\oplus_{i=0}^k A_i$ is empty if $b_k' \leq 0$ and in any case it is contained in the set of consecutive residues
\[
\binom{k+2}{2}, \; \binom{k+2}{2}+1, \; \ldots, \; \sum_{i=0}^k b_i',
\]
where the numbers $b_i'$ are defined by (2). This completes the proof.
\end{proof}

The following result of Dias da Silva and Hamidoune \cite{4} is a simple consequence of (a special case of) the above theorem.

\begin{theorem}[\cite{4}] \label{thm:DiasDaSilvaHamidoune}
Let $p$ be a prime and let $A$ be a nonempty subset of $\mathbb{Z}_p$. Let $s^{\wedge} A$ denote the set of all sums of $s$ distinct elements of $A$. Then $|s^{\wedge} A| \geq \min\{p, \; s|A| - s^2 + 1\}$.
\end{theorem}

\begin{proof}
If $|A| < s$ there is nothing to prove. Otherwise put $s = k + 1$ and apply Theorem~\ref{thm:general-distinct-residues} with $A_i = A$ for all $i$. Here $b_i' = |A| - i$ for all $0 \leq i \leq k$ and hence
\begin{align*}
|(k+1)^{\wedge} A| &= \left| \oplus_{i=0}^k A_i \right| 
\geq \min\left\{p, \; \sum_{i=0}^k (|A| - i) - \binom{k+2}{2} + 1\right\} \\
&= \min\left\{p, \; (k+1)|A| - \binom{k+1}{2} - \binom{k+2}{2} + 1\right\} \\
&= \min\left\{p, \; (k+1)|A| - (k+1)^2 + 1\right\}.
\end{align*}
\end{proof}

The case $s = 2$ of the last theorem settles a problem of Erd\H{o}s and Heilbronn. Partial results on this conjecture (before its proof in \cite{4}) had been obtained in \cite{12}, \cite{9}, \cite{13}, \cite{11}, and \cite{6}.

\section{Further Examples}
An easy application of Theorem~\ref{thm:general} is the following result, proved in \cite{1}.

\begin{proposition} \label{prop:product-not-one}
If $p$ is a prime and $A, B$ are two nonempty subsets of $\mathbb{Z}_p$, then
\[
|\{a + b : a \in A, \; b \in B, \; ab \neq 1\}| \geq \min\{p, |A| + |B| - 3\}.
\]
\end{proposition}

\begin{proof}
The proof is by applying Theorem~\ref{thm:general} with $k = 1$, $h = x_0 x_1 - 1$, $A_0 = A$, $A_1 = B$, and $m = |A| + |B| - 4$. It is also shown in \cite{1} that the above estimate is tight in all nontrivial cases.
\end{proof}

Two easy extensions of the above proposition are the following.

\begin{proposition} \label{prop:product-not-g}
If $p$ is a prime and $A_0, A_1, \ldots, A_k$ are nonempty subsets of $\mathbb{Z}_p$, then for every $g \in \mathbb{Z}_p$,
\[
\left| \left\{a_0 + \cdots + a_k : a_i \in A_i, \; \prod_{i=0}^k a_i \neq g \right\} \right| \geq \min\left\{p, \; \sum_{i=0}^k |A_i| - 2k - 1\right\}.
\]
\end{proposition}

\begin{proof}
If $g = 0$ the result follows trivially from the Cauchy-Davenport Theorem, and we thus assume that $g \neq 0$. Suppose, first, that $|A_i| > 1$ for all $i$. If $\sum_{i=0}^k |A_i| - 2k - 2 < p$ apply Theorem~\ref{thm:general} with $h = \prod_{i=0}^k x_i - g$ and $m = \sum_{i=0}^k |A_i| - 2k - 2$. Here $c_i = |A_i| - 1$ and the coefficient of $\prod_{i=0}^k x_i^{c_i}$ in $h \cdot (x_0 + \cdots + x_k)^m$ is $m! / \prod (c_i - 1)!$, which is nonzero modulo $p$, implying the desired result. Otherwise, replace some of the sets $A_i$ by nonempty subsets $A_i'$ satisfying $|A_i'| > 1$ and $\sum_{i=0}^k |A_i'| = p + 2k + 1$ and apply the result to the sets $A_i'$.

When $|A_i| = 1$ for several sets $A_i$ it is easy to deduce the result by applying the previous case to the sets $A_j$ of cardinality greater than $1$ with an appropriately modified value of $g$. We omit the details.
\end{proof}

\begin{proposition} \label{prop:pairwise-product-not-one}
If $p$ is a prime and $A_0, A_1, \ldots, A_k$ are subsets of $\mathbb{Z}_p$, where $|A_i| \geq k + 1$ for all $i$, then
\[
\left| \left\{a_0 + \cdots + a_k : a_i \in A_i, \; a_i \cdot a_j \neq 1 \text{ for all } 0 \leq i < j \leq k \right\} \right| \geq \min\left\{p, \; \sum_{i=0}^k |A_i| - (k+1)^2 + 1\right\}.
\]
\end{proposition}

\begin{proof}
If $\sum_{i=0}^k |A_i| - (k+1)^2 < p$ apply Theorem~\ref{thm:general} with $h = \prod_{0 \leq i < j \leq k} (x_i \cdot x_j - 1)$ and $m = \sum_{i=0}^k |A_i| - (k+1)^2$. Otherwise, replace some of the sets $A_i$ by nonempty subsets $A_i'$ satisfying $\sum_{i=0}^k |A_i'| = p + (k+1)^2$ and apply the result to the sets $A_i'$.
\end{proof}

\begin{remark}
The estimate in the last proposition is not sharp. In particular, it is not too difficult to show that if every $A_i$ is of cardinality greater than $2 + \log_2(k+1)$ then the set
\[
S = \{a_0 + \cdots + a_k : a_i \in A_i, \; a_i \cdot a_j \neq 1 \text{ for all } 0 \leq i < j \leq k\} \tag{3}
\]
is nonempty. In fact, the following slightly stronger result is valid.
\end{remark}

\begin{proposition} \label{prop:pairwise-product-not-one-strong}
If $p$ is a prime and $A_0, \ldots, A_k$ are subsets of $\mathbb{Z}_p \setminus \{1, -1\}$, each of cardinality $s > \log_2(k+1)$ then the set $S$ defined in (3) is nonempty. This is tight for all $s \leq (p-3)/2$, as for each such $s$ there is a collection of $2^s$ sets $A_i \subset \mathbb{Z}_p \setminus \{1, -1\}$ of cardinality $s$ each for which the set $S$ from (3) is empty.
\end{proposition}

\begin{proof}
If $s > \log_2(k+1)$, let $H$ be a random subset of $(p-1)/2$ of the elements of $\mathbb{Z}_p \setminus \{1, -1\}$ obtained by choosing, for each pair $x, 1/x \in \mathbb{Z}_p \setminus \{1, -1, 0\}$, randomly and independently, exactly one of them to be a member of $H$. In addition, add $0$ to $H$. If $A_i \cap H \neq \emptyset$ for every $i$, the desired result follows by choosing $a_i \in A_i \cap H$ and by observing that $g \cdot g' \neq 1$ for every (not necessarily distinct) $g, g' \in H$. However, for every fixed $i$, if $A_i$ contains $0$ or contains both $x$ and $1/x$ for some $x \in \mathbb{Z}_p \setminus \{1, -1, 0\}$ then certainly $A_i \cap H \neq \emptyset$. Otherwise, the probability that $A_i \cap H = \emptyset$ is precisely $2^{-s} < 1/(k+1)$ showing that with positive probability $A_i \cap H \neq \emptyset$ for all $i$, as needed.

If $s \leq (p-3)/2$ let $x_1, \ldots, x_s$ be $s$ elements in $\mathbb{Z}_p \setminus \{1, -1, 0\}$ so that the product of no two is $1$. For each of the $2^s$ vectors $\delta = (\delta_1, \ldots, \delta_s) \in \{-1, 1\}^s$ define a subset $A_\delta$ by $A_\delta = \{x_1^{\delta_1}, \ldots, x_s^{\delta_s}\}$. It is easy to see that every choice of a member from each $A_\delta$ must contain some element $x_i$ and its inverse. This completes the proof.
\end{proof}

We conclude the section with the following.

\begin{proposition} \label{prop:sum-with-conditions}
If $p$ is a prime and $A, B$ are two nonempty subsets of $\mathbb{Z}_p$, with $|A| > |B|$ then for any $e \in \mathbb{Z}_p$
\[
|\{a + b : a \in A, \; b \in B, \; ab \neq e \text{ and } a \neq b\}| \geq \min\{p, |A| + |B| - 4\}. \tag{4}
\]
\end{proposition}

\begin{proof}
If $|B| \leq 2$ and $b' \in B$, then $A$ contains a subset $A'$ of $|A| - 2$ elements which are neither $b'$ nor $e b'^{-1}$ and hence in this case
\[
|\{a + b : a \in A, \; b \in B, \; ab \neq e \text{ and } a \neq b\}| \geq |b' + A'| = |A| - 2 \geq |A| + |B| - 4,
\]
as needed. We thus assume that $|A| > |B| \geq 3$. If $|A| + |B| - 5 < p$, apply Theorem~\ref{thm:general} with $k = 1$, $h = (x_0 - x_1)(x_0 \cdot x_1 - e)$, $A_0 = A$, $A_1 = B$ and $m = |A| + |B| - 5$. Here $c_0 = |A| - 1$, $c_1 = |B| - 1$, and the coefficient of $x_0^{c_0} \cdot x_1^{c_1}$ in $h \cdot (x_0 + x_1)^m$ is
\[
\binom{m}{c_0-2} - \binom{m}{c_0-1} = \frac{m!}{(c_0-1)! (c_1-1)!} (c_0 - c_1),
\]
which is nonzero modulo $p$. If $|A| + |B| - 5 \geq p$ replace $B$ by a subset $B'$ of cardinality $p + 4 - |A| (< |A|)$ and apply the result to $A$ and $B'$ to conclude that in this case $|A + B| \geq |A + B'| = p$.
\end{proof}

\begin{remark}
The last estimate is tight for all possible cardinalities $|A| > |B| > 1$ as shown by the following example.
\[
A = \{a, a + d, a + 2d, \ldots, a + c_0 d\}, \quad B = \{a, a + d, a + 2d, \ldots, a + c_1 d\},
\]
where $a, d$ are chosen so that $a(a + d) = (a + c_0 d)(a + c_1 d) = e$. The only solution of these equations in case $c_1 = 1$ (i.e., $|B| = 2$), is $e = 0$ and $d = -a$ supplying the two sets
\[
A = \{a, 0, \ldots, -(c_0 - 1)a\}, \quad B = \{a, 0\}.
\]
If $c_1 \geq 2$ the possible solutions are given by
\[
a = \sqrt{\frac{c_0 c_1 e}{(c_0 - 1)(c_1 - 1)}}, \quad d = -\frac{(c_0 + c_1 - 1)a}{c_0 c_1}.
\]
Such a solution exists for every $e$ for which the quantity $(c_0 c_1 e)(c_0 - 1)(c_1 - 1)$ is a quadratic residue. For $|B| = 1$ the right hand side of (4) can be improved to $|A| - 2 = |A| + |B| - 3$, as explained above, and this is trivially tight.

If $|A| = |B| = s > 2$ then, by applying Proposition~\ref{prop:sum-with-conditions} to $A$ and a subset of cardinality $s - 1$ of $B$ we conclude that in this case for every $e \in \mathbb{Z}_p$
\[
|\{a + b : a \in A, \; b \in B, \; ab \neq e \text{ and } a \neq b\}| \geq \min\{p, |A| + |B| - 5\}.
\]
It is not difficult to check that if $s \leq 2$ then the set in the left hand side of the last inequality may be empty. For all $s \geq 3$ the above estimate is tight, as shown by an easy modification of the example described above.
\end{remark}

\section{Concluding Remarks and Open Problems}
\begin{enumerate}
\item All the results proved above hold for subsets of an arbitrary field of characteristic $p$ instead of $\mathbb{Z}_p$ with the same proof.

\item Theorem~\ref{thm:DiasDaSilvaHamidoune} implies that if $A$ is a subset of $\mathbb{Z}_p$ and $|A| \geq (p + s^2 - 1)/s$, then $s^{\wedge} A = \mathbb{Z}_p$. This can be used to construct certain explicit codes for write once memories, a notion introduced by Rivest and Shamir in \cite{14}. Here is a brief description of this application. Motivated by the existence of memory devices as optical disks or paper tapes that have a number of ``write once'' bits (called wits), each of which contains initially a $0$ that can be irreversibly changed to a $1$, the authors of \cite{14} considered the problem of finding efficient encoding schemes that enable one to use a small number of wits to represent and update one of $v$ possible values $t$ times. Following \cite{14} let us denote by $w(\langle v \rangle^t)$ the minimum possible number of wits needed for this task. It is shown in \cite{14} that $w(\langle v \rangle^t) = \Theta(\max\{t, \frac{t \log v}{\log t}\})$ and it is conjectured that in fact as $t$ and $v$ tend to infinity
\[
w(\langle v \rangle^t) = (1 + o(1)) \max\left\{t, \frac{t \log v}{\log t}\right\}.
\]
This conjecture is false, since it is not difficult to show that, e.g., for every fixed positive $\epsilon < 0.5$
\[
w(\langle v \rangle^{\epsilon v}) \geq 2\epsilon v.
\]

\item Lemma~\ref{lem:coefficient-vandermonde} can be extended to compute the coefficient of $\prod_{i=0}^k x_i^{c_i}$ in the polynomial
\[
(x_0 + \cdots + x_k)^m \prod_{k \geq i > j \geq 0} (x_i - x_j)^\alpha
\]
for an arbitrary positive integer $\alpha$. In particular, Dyson's conjecture (first proved by Gunson \cite{7} and Wilson \cite{18}) determines the coefficient of $\prod_{i=0}^k x_i^{|E|/2}$ for even values of $|E|$. See also \cite{15}, \cite{19} for some related results.

\item Vosper \cite{16}, \cite{17} determined all cases of equality in the Cauchy-Davenport Theorem. It would be interesting to prove an analogous result for Proposition~\ref{prop:main-example}, Theorem~\ref{thm:ErdosHeilbronn} or the results in Section~4.

\item There are numerous variants of the Cauchy-Davenport Theorem for the non-prime case, including results by Chowla, Scherk, Sheperdson, Kneser and others. See \cite{10} for many of these results. It would be interesting to obtain non-prime analogs for the results obtained here.
\end{enumerate}

\section*{Acknowledgments}
The first author would like to thank Doron Zeilberger for helpful discussions.
\begin{thebibliography}{19}

\hypertarget{bib:1}{}
\bibitem{1} N. Alon, M. B. Nathanson, and I. Z. Ruzsa. Adding distinct congruence classes modulo a prime. \textit{American Math. Monthly} 102: 250-255, 1995.

\hypertarget{bib:2}{}
\bibitem{2} N. Alon and M. Tarsi. Colorings and orientations of graphs. \textit{Combinatorica}, 12:125-134, 1992.

\hypertarget{bib:3}{}
\bibitem{3} H. Davenport, On the addition of residue classes. \textit{J. London Math. Soc.} 10: 30-32, 1935.

\hypertarget{bib:4}{}
\bibitem{4} J. A. Dias da Silva and Y. O. Hamidoune. Cyclic spaces for Grassmann derivatives and additive theory. \textit{Bull. London Math. Soc.}, 26: 140-146, 1994.

\hypertarget{bib:5}{}
\bibitem{5} P. Erdős and R. L. Graham. Old and New Problems and Results in Combinatorial Number Theory. L'Enseignement Mathématique, Geneva, 1980.

\hypertarget{bib:6}{}
\bibitem{6} G. A. Freiman, L. Low, and J. Pitman. The proof of Paul Erdős' conjecture of the addition of different residue classes modulo a prime number. In: \textit{Structure Theory of Set Addition}, June 1993, CIRM Marseille, pp. 99-108, 1993.

\hypertarget{bib:7}{}
\bibitem{7} J. Gunson, Proof of a conjecture of Dyson in the statistical theory of energy levels. \textit{J. Math. Phys.} 3: 752-753, 1962.

\hypertarget{bib:8}{}
\bibitem{8} M. P. A. Macmahon, Combinatory Analysis. Chelsea Publishing Company, 1915, Chapter V.

\hypertarget{bib:9}{}
\bibitem{9} R. Mansfield. How many slopes in a polygon? \textit{Israel J. Math.}, 39:265-272, 1981.

\hypertarget{bib:10}{}
\bibitem{10} M. B. Nathanson. Additive Number Theory: 2. Inverse Theorems and the Geometry of Sumsets. Springer-Verlag, New York, 1995.

\hypertarget{bib:11}{}
\bibitem{11} L. Pyber. On the Erdős-Heilbronn conjecture. Personal communication. 1993.

\hypertarget{bib:12}{}
\bibitem{12} U.-W. Rickert. Über eine Vermutung in der additiven Zahlentheorie. PhD thesis, Tech. Univ. Braunschweig, 1976.

\hypertarget{bib:13}{}
\bibitem{13} Ö. J. Rödseth. Sums of distinct residues mod p. \textit{Acta Arith.} 65: 181-184, 1994.

\hypertarget{bib:14}{}
\bibitem{14} R. L. Rivest and A. Shamir. How to reuse a "write once" memory. \textit{Information and Computation}, 55: 1-19, 1982.

\hypertarget{bib:15}{}
\bibitem{15} J. R. Stembridge, A short proof of Macdonald's Conjecture for the root systems of type A, \textit{Proc. AMS} 102: 777-786, 1988.

\hypertarget{bib:16}{}
\bibitem{16} A. G. Vosper. The critical pairs of subsets of a group of prime order. \textit{J. London Math. Soc.} 31: 200-205, 1956.

\hypertarget{bib:17}{}
\bibitem{17} A. G. Vosper. Addendum to "The critical pairs of subsets of a group of prime order". \textit{J. London Math. Soc.} 31: 280-282, 1956.

\hypertarget{bib:18}{}
\bibitem{18} K. Wilson. Proof of a conjecture of Dyson. \textit{J. Math. Phys.} 3: 1040-1043, 1962.

\hypertarget{bib:19}{}
\bibitem{19} D. Zeilberger. A combinatorial proof of Dyson's conjecture. \textit{Discrete Math.} 41: 317-321, 1982.

\end{thebibliography}
\end{document}