% content.tex - The Polynomial Method and Restricted Sums of Congruence Classes
% Based on Alon, Nathanson, Ruzsa (1996)

\chapter{Introduction}

\begin{theorem}[Cauchy-Davenport Theorem]
Let $p$ be a prime and $A,B$ be nonempty subsets of the cyclic group $\Zp$. Then
\[
|A+B| \geq \min\{p, |A|+|B|-1\},
\]
where $A+B = \{a+b: a\in A, b\in B\}$.
\end{theorem}

The polynomial method provides a unified approach to various problems in additive number theory.

\begin{proposition}[Proposition 1.2]
Let $p$ be a prime, and let $A_{0},A_{1},\ldots,A_{k}$ be nonempty subsets of the cyclic group $\Zp$. If $|A_{i}|\neq|A_{j}|$ for all $0\leq i<j\leq k$ and $\sum_{i=0}^{k}|A_{i}|\leq p+\binom{k+2}{2}-1$ then
\[
|\{a_{0}+a_{1}+\ldots+a_{k}:a_{i}\in A_{i},a_{i}\neq a_{j}\ \text{for all}\ i\neq j\}|\geq\sum_{i=0}^{k}|A_{i}|-\binom{k+2}{2}+1.
\]
\end{proposition}

\begin{theorem}[Theorem 1.3, Erdős-Heilbronn Conjecture]
If $p$ is a prime, and $A$ is a nonempty subset of $\Zp$, then
\[
|\{a+a^{\prime}:a,a^{\prime}\in A,a\neq a^{\prime}\}|\geq \min\{p,2|A|-3\}.
\]
\end{theorem}

\chapter{Preliminaries}

\begin{definition}[Restricted sum]
Let $p$ be a prime. For a polynomial $h=h(x_{0},x_{1},\ldots,x_{k})$ over $\Zp$ and for subsets $A_{0},A_{1},\ldots,A_{k}$ of $\Zp$, define
\[
\oplus_h\sum_{i=0}^{k}A_{i}=\{a_{0}+a_{1}+\ldots+a_{k}:a_{i}\in A_{i},\ h(a_{0},a_{1},\ldots,a_{k})\neq 0\}.
\]
\end{definition}

\begin{definition}[Distinct residues sum]
Let $p$ be a prime, and let $A_{0},A_{1},\ldots,A_{k}$ be nonempty subsets of $\Zp$. Define
\[
\oplus_{i=0}^{k}A_{i}=\{a_{0}+a_{1}+\ldots+a_{k}:a_{i}\in A_{i},a_{i}\neq a_{j}\ \text{for all}\ i\neq j\}.
\]
\end{definition}

\begin{lemma}[Combinatorial Nullstellensatz, Lemma 2.2]
Let $P=P(x_{0},x_{1},\ldots,x_{k})$ be a polynomial in $k+1$ variables over an arbitrary field $F$. Suppose that the degree of $P$ as a polynomial in $x_{i}$ is at most $c_{i}$ for $0\leq i\leq k$, and let $A_{i}\subset F$ be a set of cardinality $c_{i}+1$. If $P(x_{0},x_{1},\ldots,x_{k})=0$ for all $(k+1)$-tuples $(x_{0},\ldots,x_{k})\in A_{0}\times A_{1}\times\ldots\times A_{k}$, then $P\equiv 0$.
\end{lemma}

\chapter{General Polynomial Method Theorem}

\begin{theorem}[General Theorem 2.1]
Let $p$ be a prime and let $h=h(x_{0},\ldots,x_{k})$ be a polynomial over $\Zp$. Let $A_{0},A_{1},\ldots,A_{k}$ be nonempty subsets of $\Zp$, where $|A_{i}|=c_{i}+1$ and define $m=\sum_{i=0}^{k}c_{i}-\deg(h).$ If the coefficient of $\prod_{i=0}^{k}x_{i}^{c_{i}}$ in
\[
(x_{0}+x_{1}+\cdots+x_{k})^{m}h(x_{0},x_{1},\ldots,x_{k})
\]
is nonzero (in $\Zp$) then
\[
|\oplus_h\sum_{i=0}^{k}A_{i}|\geq m+1
\]
(and hence $m<p$).
\end{theorem}

\begin{proof}
Suppose the assertion is false, and let $E$ be a multiset of $m$ elements of $\Zp$ that contains $\oplus_h\sum_{i=0}^{k}A_{i}$. Let
\[
Q(x_{0},\ldots,x_{k})=h(x_{0},\ldots,x_{k})\cdot\prod_{e\in E}(x_{0}+\ldots+x_{k}-e).
\]
Then $Q(x_{0},\ldots,x_{k})=0$ for all $(x_{0},\ldots,x_{k})\in A_{0}\times\cdots\times A_{k}$. The degree of $Q$ is $\sum_{i=0}^{k}c_{i}$.

For each $i$, define $g_{i}(x_{i})=\prod_{a\in A_{i}}(x_{i}-a)=x_{i}^{c_{i}+1}-\sum_{j=0}^{c_{i}}b_{ij}x_{i}^{j}$. Let $\overline{Q}$ be obtained from $Q$ by replacing each $x_{i}^{c_{i}+1}$ with $\sum_{j=0}^{c_{i}}b_{ij}x_{i}^{j}$. Then $\overline{Q}$ vanishes on $A_{0}\times\cdots\times A_{k}$ and has $x_{i}$-degree at most $c_{i}$. By Lemma 2.2, $\overline{Q}\equiv 0$.

However, the coefficient of $\prod_{i=0}^{k}x_{i}^{c_{i}}$ in $\overline{Q}$ equals its coefficient in $Q$, which is nonzero by assumption. This contradiction proves the theorem.
\end{proof}

\chapter{Cauchy-Davenport Theorem Proof}

\begin{theorem}[Cauchy-Davenport Theorem Proof]
If $|A|+|B|\leq p+1$ apply Theorem 2.1 with $h\equiv 1$, $k=1$, $A_{0}=A$, $A_{1}=B$ and $m=|A|+|B|-2$. Here $c_{0}=|A|-1$, $c_{1}=|B|-1$ and the relevant coefficient is $\binom{m}{c_{0}}$ which is nonzero modulo $p$ (as $m<p$). If $|A|+|B|>p+1$ replace $B$ by a subset $B^{\prime}$ of cardinality $p+1-|A|$ and apply the result above to conclude $|A+B|\geq p$.
\end{theorem}

\chapter{Distinct Residues Sums}

\begin{lemma}[Lemma 3.1]
Let $c_{0},\ldots,c_{k}$ be nonnegative integers and suppose $\sum_{i=0}^{k}c_{i}=m+\binom{k+1}{2}$, where $m$ is a nonnegative integer. Then the coefficient of $\prod_{i=0}^{k}x_{i}^{c_{i}}$ in the polynomial
\[
(x_{0}+x_{1}+\ldots+x_{k})^{m}\prod_{k\geq i>j\geq 0}(x_{i}-x_{j})
\]
is
\[
\frac{m!}{c_{0}!c_{1}!\ldots c_{k}!}\prod_{k\geq i>j\geq 0}(c_{i}-c_{j}).
\]
\end{lemma}

\begin{proof}
The product $\prod_{k\geq i>j\geq 0}(x_{i}-x_{j})$ is the Vandermonde determinant $\det(x_{i}^{j})_{0\leq i\leq k,0\leq j\leq k}$. The result follows by combinatorial manipulation.
\end{proof}

\begin{theorem}[Proposition 1.2 Proof]
Define $h(x_{0},\ldots,x_{k})=\prod_{k\geq i>j\geq 0}(x_{i}-x_{j})$. Suppose $|A_{i}|=c_{i}+1$ and put $m=\sum_{i=0}^{k}c_{i}-\binom{k+1}{2}$. By Lemma 3.1 the coefficient of $\prod_{i=0}^{k}x_{i}^{c_{i}}$ in $h\cdot(x_{0}+\ldots+x_{k})^{m}$ is
\[
\frac{m!}{c_{0}!c_{1}!\ldots c_{k}!}\prod_{k\geq i>j\geq 0}(c_{i}-c_{j}),
\]
which is nonzero modulo $p$ since $m<p$ and the $c_{i}$ are pairwise distinct. Theorem 2.1 gives the result.
\end{theorem}

\begin{theorem}[Theorem 3.2]
Let $p$ be a prime, and let $A_{0},\ldots,A_{k}$ be nonempty subsets of $\Zp$, where $|A_{i}|=b_{i}$, and suppose $b_{0}\geq b_{1}\ldots\geq b_{k}$. Define $b^{\prime}_{0},\ldots,b^{\prime}_{k}$ by $b^{\prime}_{0}=b_{0}$ and $b^{\prime}_{i}=\min\{b^{\prime}_{i-1}-1,b_{i}\}$ for $1\leq i\leq k$. If $b^{\prime}_{k}>0$ then
\[
|\oplus_{i=0}^{k}A_{i}|\geq \min\{p,\sum_{i=0}^{k}b^{\prime}_{i}-\binom{k+2}{2}+1\}.
\]
\end{theorem}

\chapter{Further Applications}

\begin{proposition}[Proposition 4.1]
If $p$ is a prime and $A,B$ are two nonempty subsets of $\Zp$, then
\[
|\{a+b:a\in A,b\in B,ab\neq 1\}|\geq \min\{p,|A|+|B|-3\}.
\]
\end{proposition}

\begin{proposition}[Proposition 4.2]
If $p$ is a prime and $A_{0},A_{1},\ldots,A_{k}$ are nonempty subsets of $\Zp$, then for every $g\in \Zp$,
\[
|\{a_{0}+\ldots+a_{k}:a_{i}\in A_{i},\prod_{i=0}^{k}a_{i}\neq g\}|\geq \min\{p,\sum_{i=0}^{k}|A_{i}|-2k-1\}.
\]
\end{proposition}

\begin{proposition}[Proposition 4.3]
If $p$ is a prime and $A_{0},A_{1},\ldots,A_{k}$ are subsets of $\Zp$, where $|A_{i}|\geq k+1$ for all $i$, then
\[
|\{a_{0}+\ldots+a_{k}:a_{i}\in A_{i},a_{i}\cdot a_{j}\neq 1\text{ for all }0\leq i<j\leq k\}|\geq \min\{p,\sum_{i=0}^{k}|A_{i}|-(k+1)^{2}+1\}.
\]
\end{proposition}

\chapter{Concluding Remarks}

\begin{remark}
All results hold for subsets of an arbitrary field of characteristic $p$ instead of $\Zp$, with the same proof.
\end{remark}

\begin{remark}
Theorem 3.3 implies that if $A$ is a subset of $\Zp$ and $|A|\geq(p+s^{2}-1)/s$, then $s^{\wedge}A=\Zp$. This can be used to construct explicit codes for write-once memories.
\end{remark}

\begin{problem}
Determine all cases of equality in Proposition 1.2, Theorem 1.3 or the results in Section 4.
\end{problem}

\begin{problem}
Obtain non-prime analogs for the results obtained here.
\end{problem}